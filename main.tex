\documentclass{report}
\usepackage{graphicx} % Required for inserting images
\usepackage{amsmath}

\title{ \Huge Experiment9}
\author{ \Large Submitted by: Shreejal Bhattarai}
\date{November 29, 2024}
\begin{document}
\maketitle

\section{Introduction}
The purpose of this experiment is to investigate the relationships between centripetal force, mass, radius, and angular velocity for an object undergoing uniform circular motion. Specifically, the lab aims to verify the theoretical relationships through experimental measurements and data analysis. It points toward the center of the circle and is expressed as:
\begin{equation}
  { F_{c} = m \cdot a_{c} = m \cdot {\frac{v^2}{r}}}
\end{equation}
where:
\begin{itemize}
    \item m is the mass of the object
    \item v is the velocity of the object
    \item r is the radius of the circle
\end{itemize}

The tangential velocity, v is given by:
\[
v = \omega \cdot r
\]
Also, from equation 1,
\[
F_{c} = m \cdot \frac{(r \cdot \omega)^2}{r}
\]
conversely, as:

\begin{equation}
    F_{c} = m \cdot r \cdot \omega
\end{equation}

\newpage
\section{Procedure}

\subsection{Part 1: Constant Mass}
In the first part of the experiment, the apparatus is loaded with a fixed mass, which includes the mass holder. The radius of the circular motion is adjusted and must remain constant throughout the experiment. The mass is then rotated at five different angular velocities \( \omega \) to gather data on the relationship between centripetal force and angular velocity. For each trial, the centripetal force is recorded. It is important to reset the force sensor before each run to ensure accurate measurements. After the data collection, two graphs are plotted: 
The product of mass and radius \( m \cdot \omega \) will be analyzed to determine the slope, which provides insight into the relationship between centripetal force and angular velocity.

\subsection{Part 2: Constant Angular Velocity}
In the second part of the experiment, the apparatus is adjusted to a fixed angular velocity \( \omega \). The mass is varied by using different weights, allowing for an analysis of how centripetal force changes with mass while maintaining a constant angular velocity. For each trial, the centripetal force (\( F_c \)) is recorded. The slope of this graph is then analyzed to understand the relationship between centripetal force and mass, which helps to confirm theoretical predictions based on Newton's laws and the formula for centripetal force.







\newpage
\section{Analysis}
\subsection{Part 1: Constant Mass}
In Part 1 of the experiment, the plot of centripetal force (\( F_{c} \))  versus angular velocity (\( \omega \)) 

\end{document}
